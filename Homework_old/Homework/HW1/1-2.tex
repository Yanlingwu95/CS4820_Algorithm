\documentclass[12pt]{article}
%\usepackage{fullpage}
\usepackage{epic}
\usepackage{eepic}
\usepackage{paralist}
\usepackage{graphicx}
\usepackage{algorithm,algorithmic}
\usepackage{tikz}
\usepackage{xcolor,colortbl}
\usepackage{wrapfig}


%%%%%%%%%%%%%%%%%%%%%%%%%%%%%%%%%%%%%%%%%%%%%%%%%%%%%%%%%%%%%%%%
% This is FULLPAGE.STY by H.Partl, Version 2 as of 15 Dec 1988.
% Document Style Option to fill the paper just like Plain TeX.

\typeout{Style Option FULLPAGE Version 2 as of 15 Dec 1988}

\topmargin 0pt
\advance \topmargin by -\headheight
\advance \topmargin by -\headsep

\textheight 8.9in

\oddsidemargin 0pt
\evensidemargin \oddsidemargin
\marginparwidth 0.5in

\textwidth 6.5in
%%%%%%%%%%%%%%%%%%%%%%%%%%%%%%%%%%%%%%%%%%%%%%%%%%%%%%%%%%%%%%%%

\pagestyle{empty}
\setlength{\oddsidemargin}{0in}
\setlength{\topmargin}{-0.8in}
\setlength{\textwidth}{6.8in}
\setlength{\textheight}{9.5in}


\def\ind{\hspace*{0.3in}}
\def\gap{0.1in}
\def\bigap{0.25in}
\newcommand{\Xomit}[1]{}


\begin{document}

\setlength{\parindent}{0in}
\addtolength{\parskip}{0.1cm}
\setlength{\fboxrule}{.5mm}\setlength{\fboxsep}{1.2mm}
\newlength{\boxlength}\setlength{\boxlength}{\textwidth}
\addtolength{\boxlength}{-4mm}
\begin{center}\framebox{\parbox{\boxlength}{{\bf
CS 4820, Spring 2018 \hfill Homework 1, Problem 2}\\
% TODO: fill in your own name, netID, and collaborators
Name: \\
NetID: \\
Collaborators:
}}
\end{center}
\vspace{5mm}

{\bf (2)} {\em (10 points)} \\
Consider the following scenario:
$n$ students get flown out to the Bay Area
for a day of interviews at a large technology company.
The interviews are organized as follows.
There are $m$ time slots during the day, and $n$ interviewers,
where $m \ge n$.
Each student $s$ has a fixed {\em schedule} which gives, for each of the
$n$ interviewers, the time slot in which $s$ meets with that interviewer.
This, in turn, defines a schedule for each interviewer $i$,
giving the time slots in which $i$ meets each student.
The schedules have the property that
\begin{compactitem}
\item each student sees each interviewer exactly once,
\item no two students see the same interviewer in the same time slot, and
\item no two interviewers see the same student in the same time slot.
\end{compactitem}

Now, the interviewers decide that a full day of interviews like this
seems pretty tedious, so they come up with the following scheme.
Each interviewer $i$ will pick a {\em distinct} student $s$.
At the end of $i$'s scheduled meeting with $s$, $i$ will take $s$
to one of the company's many in-house eateries, 
and they'll both blow off the entire rest of the day
sipping espresso and rubbing elbows with celebrity chefs.

Specifically, the plan is for each interviewer $i$,
and his or her chosen student $s$, to
{\em truncate} their schedules at the time of their meeting;
in other words, they will follow their original schedules up to the
time slot of this meeting, and then they will cancel
all their meetings for the entire rest of the day.

The crucial thing is, the interviewers want to plan this
cooperatively so as to avoid the following {\em bad situation}:
some student $s$ whose schedule has not yet been truncated
(and so is still following his/her original schedule) shows up for an
interview with an interviewer who's already left for the day.

Give an efficient
algorithm to arrange the coordinated departures of the interviewers
and students so that this scheme works out and the {\em bad situation}
described above does not happen.

{\bf Example:} \\
\newcommand\encircle[1]{%
  \tikz[baseline=(X.base)]
    \node (X) [draw, shape=circle, inner sep=0] {\strut #1};}
\definecolor{lightgray}{gray}{0.75}
Suppose $n = 2$ and $m=4$; there are students $s_1$ and $s_2$,
and interviewers $i_1$ and $i_2$.
Suppose $s_1$ is scheduled to meet $i_1$ in slot $1$
and meet $i_2$ in slot $3$;
$s_2$ is scheduled to meet $i_1$ in slot $2$ and $i_2$ in slot $4$.
Then the only solution would be to have $i_1$ leave with $s_2$
and $i_2$ leave with $s_1$.
If we scheduled $i_1$ to leave with $s_1$,
then we'd have a bad situation in which $i_1$
has already left the building at the end of the first slot, but
$s_2$ still shows up for a meeting with $i_1$ at the
beginning of the second slot.
\begin{center}
\begin{tabular}{|c|c|c|c|c|}
\hline
 & \multicolumn{4}{c|}{Time slot} \\
\hline
 & 1 & 2 & 3 & 4 \\
\hline
$s_1$ & $i_1$ & & $i_2$ & \\
\hline
$s_2$ & & $i_1$ & & $i_2$ \\
\hline
\end{tabular}
\hfill
\begin{tabular}{|c|c|c|c|c|}
\hline
 & \multicolumn{4}{c|}{Time slot} \\
\hline
 & 1 & 2 & 3 & 4 \\
\hline
$s_1$ & $i_1$ & & \encircle{$i_2$} & \\
\hline
$s_2$ & & \encircle{$i_1$} & & $i_2$ \\
\hline
\end{tabular}
\hfill
\begin{tabular}{|c|c|c|c|c|}
\hline
 & \multicolumn{4}{c|}{Time slot} \\
\hline
 & 1 & 2 & 3 & 4 \\
\hline
$s_1$ & \encircle{$i_1$} & & $i_2$ & \\
\hline
$s_2$ & & \cellcolor{lightgray}$i_1$ & & $i_2$ \\
\hline
\end{tabular}
\end{center}

\vskip \bigap

%% Your solution goes here.

\end{document}
