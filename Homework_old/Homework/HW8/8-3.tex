\documentclass[11pt]{article}
%\usepackage{fullpage}
\usepackage{epic}
\usepackage{eepic}
\usepackage{paralist}
\usepackage{graphicx}
\usepackage{algorithm,algorithmic}
\usepackage{tikz}
\usepackage{xcolor,colortbl}
\usepackage{wrapfig}


%%%%%%%%%%%%%%%%%%%%%%%%%%%%%%%%%%%%%%%%%%%%%%%%%%%%%%%%%%%%%%%%
% This is FULLPAGE.STY by H.Partl, Version 2 as of 15 Dec 1988.
% Document Style Option to fill the paper just like Plain TeX.

\typeout{Style Option FULLPAGE Version 2 as of 15 Dec 1988}

\topmargin 0pt
\advance \topmargin by -\headheight
\advance \topmargin by -\headsep

\textheight 8.9in

\oddsidemargin 0pt
\evensidemargin \oddsidemargin
\marginparwidth 0.5in

\textwidth 6.5in
%%%%%%%%%%%%%%%%%%%%%%%%%%%%%%%%%%%%%%%%%%%%%%%%%%%%%%%%%%%%%%%%

\pagestyle{empty}
\setlength{\oddsidemargin}{0in}
\setlength{\topmargin}{-0.8in}
\setlength{\textwidth}{6.8in}
\setlength{\textheight}{9.5in}


\def\ind{\hspace*{0.3in}}
\def\gap{0.1in}
\def\bigap{0.25in}
\newcommand{\Xomit}[1]{}


\begin{document}

\setlength{\parindent}{0in}
\addtolength{\parskip}{0.1cm}
\setlength{\fboxrule}{.5mm}\setlength{\fboxsep}{1.2mm}
\newlength{\boxlength}\setlength{\boxlength}{\textwidth}
\addtolength{\boxlength}{-4mm}
\begin{center}\framebox{\parbox{\boxlength}{{\bf
CS 4820, Spring 2018 \hfill Homework 8, Problem 3}\\
% TODO: fill in your own name, netID, and collaborators
Name: \\
NetID: \\
Collaborators:
}}
\end{center}
\vspace{5mm}

{\bf (3)} {\em (12 = 2+10 points)}
If $G$ is a flow network and $f$ is a flow in $G$, 
we say that $f$ \emph{saturates} an edge $e$ if the flow value 
on that edge is equal to its capacity, i.e. $f(e) = c_e$. 
The {\sc Flow Saturation} problem is the following 
decision problem: given a flow network $G$ and a positive integer 
$k$, determine if there exists a flow $f$ in $G$ such that 
$f$ saturates at least $k$ edges of $G$.
This problem asks you to prove that {\sc Flow Saturation} is NP-complete.
\vskip \gap
{\bf (a)} {\em (2 points)}
What is wrong with the following incorrect solution?
\begin{quotation}
The {\sc Flow Saturation} problem is in NP because it has
a polynomial-time verifier that takes the pair $(G,k)$ along
with a flow $f$, and checks that $f$ satisfies conservation
and capacity constraints (in linear time).  While checking
capacity constraints it also keeps a counter of how many
edges are saturated, and it reports ``yes'' if the conservation
and capacity constraints are satisfied, and the number of 
saturated edges is at least $k$.

To prove that {\sc Flow Saturation} is NP-complete we reduce
from {\sc Hamiltonian Path}.  Given a directed graph $G_0$ that
is an instance of {\sc Hamiltonian Path}, our reduction creates
a new graph $G$ consisting of a copy of $G_0$ together with
four extra vertices $\{s,s',t',t\}$.  $G$ has two new edges $(s,s')$
and $(t',t)$, and it also has edges from $s'$ to every vertex
of $G_0$ and from every vertex of $G_0$ to $t'$.  Finally we set
all edge capacities to 1, and we treat this as an instance of
{\sc Flow Saturation} with source $s$, sink $t$, and parameter
$k=n+3$.  The reduction takes linear time: if $G_0$ has $n$
vertices and $m$ edges, then $G$ has $n+4$ vertices and 
$m+2n+2$ edges, and it takes constant time to insert each
vertex or edge into the adjacency list representation of 
$G$.

To prove the correctness of the reduction, we show the
following two statements.  

{\bf If $G_0$ has a Hamiltonian
path, then $G$ has a flow that saturates $n+3$ edges.} 
Indeed, if the Hamiltonian path $P_0$ in $G_0$ is from $s_0$
to $t_0$, then $G$ contains a path $P$ of length $n+3$ from
$s$ to $t$, that begins with $s,s',s_0$, then traverses
the entire path $P_0$ to reach $t_0$, then ends with $t_0,t',t$.
Sending one unit of flow on this path $P$ saturates all of
its $n+3$ edges.  

{\bf If $G$ has a flow that saturates 
$n+3$ edges, then $G_0$ has a Hamiltonian path.}  Indeed,
since $G$ has just a single unit-capacity edge leaving $s$,
the max-flow value in $G$ is equal to 1 and so a maximum
flow consists of a single path from $s$ to $t$.  If this 
path saturates $n+3$ edges, then two of its edges leave
$s$ and $s'$, respectively, two of its edges come into 
$t'$ and $t$, respectively, and the remaining $n-1$ edges
belong to $G_0$.  Those $n-1$ edges must form a Hamiltonian
path in $G_0$.
\end{quotation}
\vskip \gap
{\bf (b)} { \em (10 points)} 
Prove that 
the {\sc Flow Saturation} problem is indeed NP-complete.
\vskip \gap
{\bf REMARK:  Part (a) contains a valid proof that {\sc Flow
Saturation} belongs to NP.
Therefore, in doing part (b), you do not need to
include that step in your solution.}



\vskip \bigap

%% Your solution goes here.

\end{document}
