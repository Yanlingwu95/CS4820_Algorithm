\documentclass[12pt]{article}
%\usepackage{fullpage}
\usepackage{epic}
\usepackage{eepic}
\usepackage{paralist}
\usepackage{graphicx}
\usepackage{algorithm,algorithmic}
\usepackage{tikz}
\usepackage{xcolor,colortbl}
\usepackage{wrapfig}


%%%%%%%%%%%%%%%%%%%%%%%%%%%%%%%%%%%%%%%%%%%%%%%%%%%%%%%%%%%%%%%%
% This is FULLPAGE.STY by H.Partl, Version 2 as of 15 Dec 1988.
% Document Style Option to fill the paper just like Plain TeX.

\typeout{Style Option FULLPAGE Version 2 as of 15 Dec 1988}

\topmargin 0pt
\advance \topmargin by -\headheight
\advance \topmargin by -\headsep

\textheight 8.9in

\oddsidemargin 0pt
\evensidemargin \oddsidemargin
\marginparwidth 0.5in

\textwidth 6.5in
%%%%%%%%%%%%%%%%%%%%%%%%%%%%%%%%%%%%%%%%%%%%%%%%%%%%%%%%%%%%%%%%

\pagestyle{empty}
\setlength{\oddsidemargin}{0in}
\setlength{\topmargin}{-0.8in}
\setlength{\textwidth}{6.8in}
\setlength{\textheight}{9.5in}


\def\ind{\hspace*{0.3in}}
\def\gap{0.1in}
\def\bigap{0.25in}
\newcommand{\Xomit}[1]{}


\begin{document}

\setlength{\parindent}{0in}
\addtolength{\parskip}{0.1cm}
\setlength{\fboxrule}{.5mm}\setlength{\fboxsep}{1.2mm}
\newlength{\boxlength}\setlength{\boxlength}{\textwidth}
\addtolength{\boxlength}{-4mm}
\begin{center}\framebox{\parbox{\boxlength}{{\bf
CS 4820, Spring 2019 \hfill Homework 2, Problem 2}\\
% TODO: fill in your own name, netID, and collaborators
Name: \\
NetID: \\
Collaborators:
}}
\end{center}
\vspace{5mm}

\vskip \bigap
{\bf (2)} {\em (15 points)}
  Consider the following simplified model of how a law enforcement
  organization such as the FBI apprehends the members of an
  organized crime ring. The crime ring has $n$ members, denoted by
  $x_1,\ldots,x_n$. A {\em law enforcement plan} is a sequence of
  $n$ actions, each of which is either:
  \begin{itemize}
    \item
      apprehending a member $x_j$ directly: this succeeds with
      probability $q_j$; or
    \item
      apprehending a member $x_j$ using another member, $x_i$, as a decoy:
      this action can only be taken if $x_i$ was already apprehended in a
      previous step. The probability of success is $p_{ij}$.
  \end{itemize}
  Let's assume that, if an attempt to apprehend $x_j$ fails, then $x_j$
  will go into hiding in a country that doesn't allow extradition, and
  hence $x_j$ can never be apprehended after a failed attempt.
  Therefore, a law enforcement plan is only considered {\em valid} if
  for each of the crime ring's $n$ members, the plan contains only one
  attempt to apprehend him or her.

  Assume we are given an input that specifies the number of members
  in the crime ring, $n$, and the probabilities $q_j$ and $p_{ij}$
  for each $i \ne j$. You can assume these numbers are strictly
  positive and that $p_{ij} = p_{ji}$  for all $i \ne j$.
  \vskip \gap
{\bf (2a)} {\em (5 points)}
  Design an algorithm to compute a valid law enforcement plan
  that maximizes the probability of apprehending all of the
  crime ring's members, {\em in the order $x_1,x_2,\ldots,x_n$}.
  In other words, for this part of the problem you should assume
  that the $j^{\mathrm{th}}$ action in the sequence should be
  an attempt to apprehend $x_j$, and the only thing your algorithm
  needs to decide is whether to apprehend $x_j$ directly or to use
  one of the earlier members as a decoy, and if so, which decoy to
  use.
  \vskip \gap
{\bf (2b)} {\em (10 points)}
  Design an algorithm to compute a valid law enforcement plan
  that maximizes the probability of apprehending all of the
  crime ring's members, {\em in any order}. In other words,
  for this part of the question, your algorithm must decide
  on the order in which to apprehend the crime ring's members
  {\em and} the sequence of operations to use to apprehend
  them in that order.

\vskip \bigap

%% Your solution goes here.

\end{document}
